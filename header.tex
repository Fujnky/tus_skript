\documentclass[
  bibliography=totoc,     % Literatur im Inhaltsverzeichnis
  captions=tableheading,  % Tabellenüberschriften
  titlepage=firstiscover, % Titelseite ist Deckblatt
]{scrartcl}

% Paket float verbessern
\usepackage{scrhack}

% Seitenränder etc.
\usepackage[a4paper]{geometry}

% Warnung, falls nochmal kompiliert werden muss
\usepackage[aux]{rerunfilecheck}

% deutsche Spracheinstellungen
\usepackage{polyglossia}
\setmainlanguage{german}

%\usepackage{sourceserifpro}
%\usepackage{sourcesanspro}
%\usepackage{charter}
%\usepackage{mathpazo}

% unverzichtbare Mathe-Befehle
\usepackage{amsmath}
% viele Mathe-Symbole
\usepackage{amssymb}
% Erweiterungen für amsmath
\usepackage{mathtools}

\ExplSyntaxOn
\NewDocumentCommand{\kf}{}{
\begin{align}
    \Aboxed{k\text{-Form} \ \hat= \ \text{Kuchenform}} \\
\end{align}
}
\ExplSyntaxOff

% Fonteinstellungen
\usepackage{fontspec}
% Latin Modern Fonts werden automatisch geladen

\usepackage[
  math-style=ISO,    % ┐
  bold-style=ISO,    % │
  sans-style=italic, % │ ISO-Standard folgen
  nabla=upright,     % │
  partial=upright,   % ┘
  warnings-off={           % ┐
    mathtools-colon,       % │ unnötige Warnungen ausschalten
    mathtools-overbracket, % │
  },                       % ┘
]{unicode-math}


% traditionelle Fonts für Mathematik
%\setmainfont{SourceSerifPro-Regular.otf}
%\setsansfont{SourceSansPro-Bold.otf}
%\setmathrm{SourceSansPro-Regular.otf}
%\setmathfont{Latin Modern Math}
%\setmathfont{xits-math.otf}[range={scr, bfscr}]
%\setmathfont{XITS Math}[range={cal, bfcal}, StylisticSet=1]

% Zahlen und Einheiten
\usepackage[
  locale=DE,                 % deutsche Einstellungen
  separate-uncertainty=true, % immer Fehler mit \pm
  per-mode=reciprocal,       % ^-1 für inverse Einheiten
%  output-decimal-marker=.,   % . statt , für Dezimalzahlen
]{siunitx}

% chemische Formeln
\usepackage[
  version=4,
  math-greek=default, % ┐ mit unicode-math zusammenarbeiten
  text-greek=default, % ┘
]{mhchem}

% richtige Anführungszeichen
\usepackage[autostyle]{csquotes}

% schöne Brüche im Text
\usepackage{xfrac}

% Standardplatzierung für Floats einstellen
\usepackage{float}
\floatplacement{figure}{htbp}
\floatplacement{table}{htbp}

% Floats innerhalb einer Section halten
\usepackage[
  section, % Floats innerhalb der Section halten
  below,   % unterhalb der Section aber auf der selben Seite ist ok
]{placeins}

% Captions schöner machen.
\usepackage[
  labelfont=bf,        % Tabelle x: Abbildung y: ist jetzt fett
  font=small,          % Schrift etwas kleiner als Dokument
  width=0.9\textwidth, % maximale Breite einer Caption schmaler
]{caption}
% subfigure, subtable, subref
\usepackage{subcaption}

% Grafiken können eingebunden werden
\usepackage{graphicx}
% größere Variation von Dateinamen möglich
\usepackage{grffile}

% schöne Tabellen
\usepackage{booktabs}

% Verbesserungen am Schriftbild
\usepackage{microtype}

% Literaturverzeichnis
%\usepackage[
%  backend=biber,
%  sorting=none
%]{biblatex}

% Hyperlinks im Dokument
\usepackage[
  unicode,        % Unicode in PDF-Attributen erlauben
  pdfusetitle,    % Titel, Autoren und Datum als PDF-Attribute
  pdfcreator={},  % ┐ PDF-Attribute säubern
  pdfproducer={}, % ┘
]{hyperref}
% erweiterte Bookmarks im PDF
\usepackage{bookmark}

% Trennung von Wörtern mit Strichen
\usepackage[shortcuts]{extdash}


\usepackage{tikz}
\usetikzlibrary{shapes,arrows}


\usepackage{braket}
\DeclareMathOperator{\Tr}{Tr}
\DeclareMathOperator{\Var}{Var}
\DeclareMathOperator{\I}{I}
\DeclareMathOperator{\Li}{Li}
\newcommand*\kB{k_\mathrm{B}}
\newcommand*\dif{\mathop{}\!\mathrm{d}}

\ExplSyntaxOn
\NewDocumentCommand \pdif {mmo}{
    \IfNoValueTF {#3} {
        \frac{\partial #1}{\partial #2}
    }{
        \left(\frac{\partial #1}{\partial #2}\right)\sb{#3}
    }
}

\NewDocumentCommand \tdif {mm}{\frac{\dif #1}{\dif #2}}


\NewDocumentCommand{\underarrow}{mm}{
  \underset{\makebox[0pt]{\begin{tabular}{@{}c@{}}\ensuremath{\uparrow}\\[0pt] $\small #2$ \end{tabular}}}{#1}
}
\ExplSyntaxOff


\NewDocumentCommand{\cluster}{ooo}{
    \IfNoValueTF{#3}{
        \IfNoValueTF{#2}{
            \IfNoValueTF{#1}{
                \tikz[baseline=(char.base)]{\node[shape=circle,draw,minimum size=4.5mm] (char) {};}
            }{
                \tikz[baseline=(char.base)]{\node[shape=circle,draw,inner sep=0.5mm,minimum size=4.5mm] (char) {#1};}
            }
        }{
            \tikz[baseline=(char.base)]{
            \draw
                (0,0) node[circle, draw,inner sep=0.5mm,minimum size=4.5mm] {#1}
             -- (0.75,0) node[circle, draw,inner sep=0.5mm,minimum size=4.5mm] {#2};
            }
        }
    }{
        \tikz[baseline=(char.base)]{
            \draw
                (0,0) node[circle, draw,inner sep=0.5mm,minimum size=4.5mm] {#1}
             -- (0.75,0) node[circle, draw,inner sep=0.5mm,minimum size=4.5mm] {#2}
             -- (1.5,0) node[circle, draw,inner sep=0.5mm,minimum size=4.5mm] {#3};
            }
    }
}

\NewDocumentCommand{\clustertri}{mmm}{
    \tikz[baseline=(char.base)]{
        \draw
            (0,0) node[circle, draw,inner sep=0.5mm,minimum size=4.5mm] {#1}
         -- (0.375,0.6) node[circle, draw,inner sep=0.5mm,minimum size=4.5mm] {#2}
         -- (0.75,0) node[circle, draw,inner sep=0.5mm,minimum size=4.5mm] {#3}
         -- cycle;
        }
}

\DeclareMathOperator{\e}{e}
\usepackage[makeroom]{cancel}
\numberwithin{equation}{subsection}
\usepackage{enumerate}
\newcommand*\circled[1]{\tikz[baseline=(char.base)]{
            \node[shape=circle,draw,inner sep=2pt] (char) {#1};}}
            
\def\tensor#1{\underline{\underline{#1}}}

%Tiff?